% !TEX root = ./pkMain.tex
\mode*

Lange Zeit nahm man an, dass man aufgrund der Konzentration eines Medikamentes im Blut, die Wirkung absch�tzen kann. Von der intraven�sen Injektion erwartete man resp. man nahm an, dass die Wirkung sofort eintreten w�rde.

\begin{frame}<presentation>[label=ZeitBez]
\begin{tikzpicture}[remember picture, overlay]
\node[inner sep=0pt,xshift=0cm] at (current page.center){
\tikz \draw[step=2mm,black!50] (0,0) grid (126mm,94mm);
};
{
\node[coordinate] at (1cm,-3.55cm) (start) {};
\node[coordinate] at (1cm,3cm) (end) {};
\node[coordinate] at (10cm,-3.55cm) (endy) {};
\draw[->,thick] (start)--(end);
\draw[->,thick] (start)--(endy);
};
\end{tikzpicture}

\frametitle{Zeitliche Beziehung: Plasma Konzentration $\Leftrightarrow$ Wirkung}
\note<1>{
\begin{itemize}
\item
Bolus Gabe
\item
Darstellen, dass Konzentration zu Beginn hoch, Wirkung tief.
\item
Dieselbe Wirkung bei verschiedenen Konzentrationen
\item
Keine eindeutige Beziehung zw. Konzentration und Wirkung.
\item
Infusion: Verlauf er Konzentration\\
Um so weniger zwischen Infusionsrate und Wirkung
\end{itemize}
}
\infina
\end{frame}

\mode<article>{
\begin{center}
\includeslide[width=0.8\textwidth]{ZeitBez}
\end{center}
}


\begin{frame}
\frametitle{Konzentration am Wirkort}
\begin{itemize}[<+->]
\item
Die Blutkonzentration hat keine direkte Beziehung zur wirkung
\item
Das Medikamente am Wirkort (Wirkortkonzentration) ist verantwortlich f�r Wirkung
\item
Im \enquote{Steady State} sind Konzentration im Blut und Wirkort Konzentration gleich.
\item
Die Wirkortkonzentration steigt und f\"allt entsprechend dem Konzentrationsgradienten. ($=$ Unterschied der Konzentrationen)
\end{itemize}
\note<1>{
\begin{itemize}
\item
Mit Cylinder das Effect Kompartiment zeigen.
\item
Simulation mit einem Bolus
\item
Simulation mit konstanter Infusion
\item
Simulation die zeigt, dass mit Bolus die Plasmakonzentration \enquote{\"uberschossen} werden muss.
\end{itemize}
}
\infina
\end{frame}


\mode<article>{Bei der Beschreibung der Pharmakokinetik wurde auf die
Konzentration des Medikamentes im Blut resp. Plasma Bezug genommen. Die
An�sthetika entfalten ihre Wirkung aber nicht im direkt im Blut in einem
anderen Kompartiment z.B. im Gehirn. Es braucht aber Zeit bis das
Medikament vom Blut in dieses Kompartiment aufgenommen ist. Es gibt also
ein weiteres Kompartiment, den Wirkort, das wir in unsere �berlegungen
einbeziehen m�ssen.}


\frame{
\frametitle{Input, Konzentration, Wirkung}
\mode<presentation>{
\begin{center}
    \includegraphics<1>[clip=true,width=0.9\textwidth]{./receptor/receptorOn1.png}
    \includegraphics<2>[clip=true,width=0.9\textwidth]{./receptor/receptorOn2.png}
    \includegraphics<3>[clip=true,width=0.9\textwidth]{./receptor/receptorOn3.png}
    \includegraphics<4>[clip=true,width=0.9\textwidth]{./receptor/receptorOn4.png}
    \includegraphics<5>[clip=true,width=0.9\textwidth]{./receptor/receptorOn5a.png}
    \includegraphics<6>[clip=true,width=0.9\textwidth]{./receptor/receptorOn5.png}
\end{center}
}

\mode<article>{
\centering
\begin{figure}[H]
    \includegraphics<1>[clip=true,width=0.3\textwidth]{./receptor/receptorOn3.png}
\caption{Es braucht Zeit bis sich die Wirkung, die mit einer bestimmten
Plasmakonzentration korreliert sich etabliert hat. Initial hohe Konzentration im Blut, das
Medikament hat den Rezeptor noch nicht erreicht}
\end{figure}
}
\note<1>{.}
}


\frame{
\frametitle{Wirkverlust $\Leftarrow$ Konzentration}
\mode<presentation>{
\begin{center}
    \includegraphics<1>[clip=true,width=0.9\textwidth]{./receptor/receptorOff1.png}
    \includegraphics<2>[clip=true,width=0.9\textwidth]{./receptor/receptorOff2.png}
    \includegraphics<3>[clip=true,width=0.9\textwidth]{./receptor/receptorOff3.png}
    \includegraphics<4>[clip=true,width=0.9\textwidth]{./receptor/receptorOff4.png}
    \includegraphics<5>[clip=true,width=0.9\textwidth]{./receptor/receptorOff5.png}
    \includegraphics<6>[clip=true,width=0.9\textwidth]{./receptor/receptorOff6.png}
    \includegraphics<7>[clip=true,width=0.9\textwidth]{./receptor/receptorOff7.png}
\end{center}
}
\mode<article>{
\centering
\begin{figure}[H]
    \includegraphics<1>[clip=true,width=0.3\textwidth]{./receptor/receptorOff3.png}
\caption{Die Konzentration im BLut sinkt. Dadurch entsteht ein umgekehrter Konzentrationsgradient.}
\end{figure}
}
\note<1>{.}
}








\subsection{$t_{peak}$}


\begin{frame}<presentation>[label=PeakNachBolus]
\begin{tikzpicture}[remember picture, overlay]
\node[inner sep=0pt,xshift=0cm] at (current page.center){
\tikz \draw[step=2mm,black!50] (0,0) grid (126mm,94mm);
};
{
\node[coordinate] at (1cm,-3.55cm) (start) {};
\node[coordinate] at (1cm,3cm) (end) {};
\node[coordinate] at (10cm,-3.55cm) (endy) {};
\draw[->,thick] (start)--(end);
\draw[->,thick] (start)--(endy);
};
\end{tikzpicture}

\frametitle{Maximale Konzentration nach Bolus \enquote{Peak} Konzentration}
\note<1>{
\begin{itemize}
\item
Kurve f\"ur Bolus  - $C_{e}$
\item
Zeigen der $t_{peak}$; Wenn Bolus verdoppelt wie lange geht es bis maximale Konzentration erreicht?
\item
Zeigen, dass mit doppeltem Bolus, gleiche Kurve aber doppelt so hoch
\item
Ist der Wirkeintritt bei Verdoppelung der Konzentration schneller? (Esmeron)
\item
Zeigen, dass Wirkeintritt, definiert als Zeit bis zu einer Konzentration schneller.
\item
Noch einmal mit Cylinders zeigen
\item
FentanaBolus.xls
\end{itemize}
}
\infina
\end{frame}

\mode<article>{
\begin{center}
    \includeslide[width=0.8\textwidth]{PeakNachBolus}
\end{center}
}

Nach dem Verabreichen eines Bolus eines An�sthetikums f�llt die initial hohe Konzentration im Blut ab und die Konzentration am Wirkort steigt auf Grund des Konzentrationsgradienten an. Wenn die Konzentration am Wirkort und die Konzentration im Blut gleich sind, ist die Konzentration am Wirkort maximal. Die Zeit bis diese maximale Konzentration erreicht wird ist unabh�ngig von der Gr�sse des Bolus gleich!


Warum verabreichen wir eine h�here Dosis des Nicht-Depolarisierenden Muskelrelaxans bei einer Notfalleinleitung?

\frame<presentation>[label=ZeitWirkung]{
\begin{tikzpicture}[remember picture, overlay]
\node[inner sep=0pt,xshift=0cm] at (current page.center){
\tikz \draw[step=2mm,black!50] (0,0) grid (126mm,94mm);
};
{
\node[coordinate] at (1cm,-3.55cm) (start) {};
\node[coordinate] at (1cm,3cm) (end) {};
\node[coordinate] at (10cm,-3.55cm) (endy) {};
\draw[->,thick] (start)--(end);
\draw[->,thick] (start)--(endy);
};
\end{tikzpicture}

\frametitle{Zeit $\Leftrightarrow$ Wirkung: Erh\"ohung der Dosis}
\note<1>{
\begin{itemize}
\item
Submaximale Konzentration mit geringen Bolus
\item
Zweite submaximale Dosis
\item
Dritte supramaximale Dosis
\item
Vierte supramaximale Dosis: Maximale Wirkung gleich: $t_{peak}$ nicht erkennbar.
\item
Zusammenfassen: Mit derselben Infusionsrate k\"onnen die Konzentrationen verschieden sein. F\"ur die $C_{e}$  gilt dies sogar noch mehr. (RiseToSS.xls zeigt den Verlauf bei konstanter Infusion.)
\end{itemize}
}
\infina
}

\mode<article>{
\begin{center}
\includeslide[width=0.8\textwidth]{ZeitWirkung}
\end{center}
}


\mode<article>{
\vspace{0.5cm}
\begin{minipage}{\textwidth}
}

\begin{frame}
\frametitle{T peak, Pentothal}
\begin{center}
\includegraphics[width=0.6\textwidth]{../Figures/FlaishonPentoBIS1997.jpg}
\end{center}
\note<1>{.}
\end{frame}

\mode<article>{
\end{minipage}
}

\mode<article>{
\begin{minipage}{\textwidth}
}

\begin{frame}
\frametitle{T peak, Propofol}
\begin{center}
\includegraphics[width=0.6\textwidth]{../Figures/FlaishonPropofolBIS1997.jpg}
\end{center}
\note<1>{.}
\end{frame}

\mode<article>{

Obwohl Propofol eine \enquote{schnellere} Kinetik hat, dauert es nach einer �quipotenten potenten Dosis l�nger bis ein Patient aufwacht als nach Thiopental
\vspace{.5cm}
\end{minipage}
}

\mode<article>{
\begin{minipage}{\textwidth}
}

\begin{frame}
\frametitle{T peak, Etomidate versus Propofol}
\begin{center}
\includegraphics[width=0.6\textwidth]{../Figures/EtomidatePropofolTPeak.jpg}
\end{center}
\note<1>{.}
\end{frame}

\mode<article>{
Mit Etomidate wird der maximale Effekt nach einer Bolus Dosis schneller erreicht als nach Gabe von Propofol.

\end{minipage}
}
