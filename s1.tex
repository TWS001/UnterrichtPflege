% !TEX root = ./pkMain.tex
\mode*


\begin{frame}[<+->]
\frametitle{Pharmakokinetik, Pharmakodynamik}
\begin{itemize}
\item
Pharmakokinetik beschreibt was der K�rper mit dem Medikament macht.
\mode<article>{\newline Aufnahme, Verteilung und Elimination des Medikamentes. Beschreibung des zeitlichen Verlaufs der Konzentration.}
\item
Pharmakodynamik beschreibt was das Medikament mit dem K�rper macht
\mode<article>{\newline Beziehung zwischen der (Wirkort-) Konzentration und der Wirkung.}
\end{itemize}

\note<1>{
\begin{itemize}
\item
Beschreiben der zwei S�tze!
\item
Zeichnen: Blut - Organe - Umverteilung - Wirkort - Wirkung
\item
Wenn ein Medikament i.v. verabreicht wird, wird es im Blut verteilt, in verschiedene Organe verteilt und in einigen Organen metabolisiert. Der K�rper \enquote{macht} etwas mit dem Medikament.
\item
Am Wirkort entfaltet das Medikament seine Wirkung (Rezeptor). Das Medikament \enquote{macht} etwas mit dem K�rper.
\item
Cylinder: Kinetik - Input: H�he des zentralen Kompartiments entspricht Konzentration, ist abh�ngig was der K�rper mit dem Input macht.
\end{itemize}
}

\infina
\end{frame}



