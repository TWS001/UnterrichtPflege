% !TEX root = ./pkMain.tex
\mode*

{Die Medikamenten Konzentration im Blut f�llt ab, weil das Medikament (aus dem K�rper) eliminiert, metabolisiert und umverteilt wird. {\it{Elimination}} kann aber im pharmakokinetischen Sinne auch allgemein den \enquote{Abfall} der Konzentration beschreiben. Es gibt verschiedene Begriffe mit denen die Elimination quantifiziert wird: Halbwertszeit, Clearance, Eliminationsrate. Keiner dieser Begriffe beschreibt alleine gen�gend ob die Konzentration in einer gegeben Situation schnell oder langsam abf�llt. Ein sehr wichtiger, die Elimination beschreibender Prozess ist die {\it{Clearance}}. Die {\it{Kontext sensitiven Konzentrationsabfallzeiten}} (siehe weiter hinten) beschreiben den Konzentrationsabfall resp. die Elimination umfassend.

\begin{frame}
\frametitle{Elimination, Grunds\"atzliches}
\begin{itemize}[<+->]
\item
Eliminationsrate $=$ Menge die pro Zeit eliminiert wird
\item
Eliminationsrate abh\"angig von der Konzentration


\mode<article>{Dies gilt f�r (alle) An�sthetika. Im Rahmen von klinisch \enquote{sinnvollen} Konzentrationen ist die Kapazit�t der Eliminationsprozesse nicht ausgesch�pft. Deshalb wird mehr eliminiert, wenn die Konzentration h�her ist. Man spricht in diesem Falle von {\it linearer} Kinetik. 

Es gibt auch {\it nicht lineare} Kinetik! Wie w�rde die Konzentration bei \emph{nicht linearer} Kinetik (Eliminationsrate unabh�ngig von Konzentration!) abfallen? Beispiel?}
\item
Achtung: Die {\it Zufuhrrate} ist mit einer konstanten Infusion konstant! Bei linearer Kinetik ist die Elimination aber abh�ngig von der Konzentration!
\end{itemize}
\note<1>{
\begin{itemize}
\item
...
\end{itemize}
}
\infina
\end{frame}




\begin{frame}
\frametitle{Clearance: Volumen pro Zeit, z.B. $\frac{ml}{min}$}
\begin{itemize}[<+->]
\item
Beschreibt Volumen das pro Zeit \enquote{gereinigt} wird.
\item
Proportionalit�tskonstante: Eliminationsrate in Beziehung zu Konzentration.
\item
Wenn Konzentration hoch: Eliminationsrate hoch
\item
Wie hoch kann bei einer konstanten Zufuhr (Infusionsrate) die Elimininationsrate maximal werden? (Zufuhr und Elimination finden gleichseitig statt.)
\end{itemize}

\note<1>{
\begin{itemize}
\item
Quadrat mit 3 er Unterteilung
\item
Jedes Teilvolumen 1 l, Totalvolumen $=$ 9 l
\item
Clearance ist 1l / min
\item
Wie lange geht es bis das ganze Volumen gereinigt.
\item
Simulation mit Cylinder: Dicke der Pfeile zeigen.
\end{itemize}
}
\infina
\end{frame}
