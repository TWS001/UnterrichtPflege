% !TEX root = ./pkMain.tex
\mode*


\begin{frame}
\frametitle{Begriffe, Definitionen}
\begin{itemize}
\item<1->
Bolus (Menge, \enquote{Dosis}, z.B. \enquote{\emph{mg}})


\mode<article>{Dosis bezieht sich auf die Menge (Gewicht) der wirksamen Substanz. Achtung die injzierbaren Medikamente sind in einer Fl�ssigkeit gel�st.}
\item
Konzentration: Menge pro Volumen (\enquote{\emph{$\frac{mg}{l}$}})

\mode<article>{Konzentration des Medikamentes im Blut, am Wirkort. Direkte Beziehung zwischen der Konzentration am Wirkort und der Wirkung.}
\item<2->
Infusionsrate (Menge pro Zeit, Infusionsgeschwindigkeit, z.B. \enquote{\emph{$\frac{mg}{h}$}})


\mode<article>{Menge pro Zeit. Die Zeit ist auch bei Bolusgaben zu ber�cksichtigen. Je h�ufiger die Bolusgabe repetiert wird, desto h�her wird die erreichte Konzentration sein. Auch bei repetitiven Bolusgaben wird eine \enquote{Steady State} Konzentration erreicht.}

\end{itemize}
%%%
\note<1>{
\begin{itemize}
\item
Menge in ein Volumen: Substanz verteilt sich homogen $\Rightarrow$ Konzentration. V=10L, 100 mg? 200 mg? - Umgekehrt: 500 mg, Konzentration: 50 mg/ L - Vol?; Zeigen: Nicht homogene Verteilung: sehr grosse VertVol.
Scheinbares Verteilungsvolumen! Rechnerisch!
\item
Hohe Konzentration $=$ hohe Wirkung. (Organ badet im Medikament das im Blut gel�st ist.)
\item
Dosis resp. Infusionsrate $\neq$ Konzentration! (Cylinder: Custom 10 L, IR 10;; 10,30,100 ... dann Fentanyl: IR 2)
\item
Konzentrationsgradient beachten! Nach Stop, in welche Richtung bewegt sich das Medikament?
\end{itemize}
}
\note<2>{
\begin{itemize}
\item
Infusionsgeschwindigkeit: Volumen pro Zeit, Verd\"unnung des Medikamentes, weil Menge pro Zeit wichtig ist.\\
Sie geben 2\% Propofol mit Rate 6mg/kg/h -nach dem Wechsel auf 1\% Propofol wieviel m�ssen sie geben?
\item
Bolus: Je nach zeitl. Beziehung andere Konzentration - Wirkung
\end{itemize}
}
%%%
\infina
\end{frame}

