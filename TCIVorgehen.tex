
\subsection{i.v. An�sthesie mit TCI}


\begin{frame}
\frametitle{Indikation \mode<presentation>{f�r \subsecname}}
\begin{itemize}
\item
  grunds�tzlich immer m�glich
\item
  bei hohem Risiko f�r PONV
\item
  bei maligner Hyperthermie
\item
  um Umgebungskontamination mit Volatilen zu verhindern
\item
  bei Neuromonitoring mit evozierten Potentialen
\item
  wenn nichthypnotische Eigenschaften von Propofol erw�nscht sind
\end{itemize}
\note<1>{.}
\end{frame}


\begin{frame}
\frametitle{Kontraindikationen \mode<presentation>{f�r \subsecname}}
\begin{itemize}
\item
  Absolut: Allergie gegen Propofol oder Remifentanil
\item
  Relativ: Hypovol�mie, Kreislaufinstabilit�t, Venenpunktionsstelle
  nicht einsehbar
\end{itemize}
\note<1>{.}
\end{frame}


\begin{frame}
\frametitle{Prinzip der An�sthesief�hrung mit TCI}
\begin{itemize}
\item
  mit Propofol sicherstellen, dass der Patient schl�ft.
\item
  mit Remifentanil (und Fentanyl) die schmerzbedingte Kreislaufreaktion
  behandeln
\item
  Hohe Opiat-Konzentrationen reduzieren die Wahrscheinlichkeit von
  motorischen Reaktionen auf Stimuli. (auch bei tiefen Konzentrationen
  von Propofol)
\item
  Gegen Ende der An�sthesie wird unter Ber�cksichtigung der 70\%
  Konzentrationsabfall -Zeit (bezieht sich auf durchschnittliche
  intraoperative Konzentration.) Propofol rechtzeitig abgestellt.
  Gleichzeitig wird die Remifentanil Konzentration kontinuierlich erh�ht
  (verhindern mot. Reaktionen)
\end{itemize}
\note<1>{.}
\end{frame}


\begin{frame}
\frametitle{Dosierung / Zielkonzentrationen Propofol}
\begin{itemize}
\item
  grunds�tzlich die Wirkung eintitrieren $\Rightarrow$ sehr vorsichtig
  bei �lteren Patienten
 \item
   Bei Patienten \textgreater{} 65 Jahre (biologisch) mit $2 \mu g/ml$
   Wirkortkonzentration ($C_e$) beginnen (entspricht ca. 0.5 mg Bolus). Warten bis
   $C_e$ erreicht, erst dann Konzentration
   erh�hen
\item
  Junge Patienten brauchen f�r Einlage der LM oft relativ hohe
  Konzentrationen von Propofol (6--8 $\mu g/ml$)
\item
  Intraoperative Propofolgabe gem�ss BIS (40--60) oder klinischen
  Zeichen der Wachheit (in erster Linie Reaktion auf Ansprechen)
\end{itemize}
\note<1>{.}
\end{frame}


\begin{frame}
\frametitle{Dosierung / Zielkonzentrationen, Remifentanil - Fentanyl}
\begin{itemize}
\item
  vor Einleitung 200 $\mu g$ Fentanyl (alte Patienten, kurze Eingriffe: Dosis
  reduzieren)
\item
  f�r kurze Eingriffe \textless{} 1--2 h kein zus�tzliches Fentanyl vor
  Schnitt
\item
  f�r Eingriff \textgreater{} 1--2 h in der Regel 100 - 200 $\mu g$ Fentanyl
  zus�tzlich vor Schnitt.
\item
  bei langen Operationen zu Beginn Kreislaufreaktionen (Hypertension,
  Tachykardie) mit Fentanyl behandeln (bei bariatrischen Eingriffen 1
  mg w�hrend ersten 45 min m�glich)
\item
  KEIN Fentanyl mehr 1--2 h vor Ende der Operation! - grunds�tzlich
  Fentanyl nur zu Beginn
  Intraoperativ abnehmende Fentanyl Wirkung mit Remifentanil
  kompensieren (Intraoperativ bis ca. 4--6 ng /ml); Gegen Ende der
  Operation v.a. wenn Propofol gestoppt ist, Zielkonzentration bis
  \textgreater{} 10 ng/ml. Stoppen der Infusion, wenn Haut geschlossen.
\end{itemize}
\note<1>{.}
\end{frame}



